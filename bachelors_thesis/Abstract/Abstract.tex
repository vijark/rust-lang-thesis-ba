\begin{abstract}
\begin{center}
\Huge
\emph{\textbf{Zusammenfassung/Abstract}}
\end{center}
\normalsize
\vspace{15mm}
\textit{Beim Programmieren mit einer systemnahen Sprache besteht oft das Problem, dass viele Fehler nicht erkannt werden. Solche Fehler können beim Programmierer unbemerkt bleiben, da sie meist nur schwer zu finden sind. Eine Sprache, die dem Programmierer Vertrauen beim Entwickeln gibt, hilft dabei, kritische Fehler frühzeitig zu entdecken. Dieses Vertrauen gibt ein strenger Compiler, welcher alle Speicheradressen auf Gültigkeit überprüfen kann.}

\textit{In dieser Arbeit wird die Programmierung mit der Sprache Rust gezeigt, mit den Eigenheiten, die dafür sorgen, dass Speicherfehler bereits beim Kompilieren gefunden werden können. Zudem werden Unterschiede zu C/C++ aufgezeigt.}
\end{abstract}
