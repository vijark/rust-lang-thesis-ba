%KOMA-Script-Klasse: scrreprt
%deutsches Design, Schriftgröße 12, DIN A4
\documentclass[12pt,a4paper,oneside]{scrreprt}
%Seitenspiegel einstellen
\usepackage[a4paper]{geometry}
\geometry{a4paper,left=30mm,right=25mm,
bottom=20mm,top=15mm,bindingoffset=2mm,
includehead,includefoot}



%schalte Umlaute frei
\usepackage[ngerman]{babel}
%passende Codierung
\usepackage[utf8]{inputenc}
%Seitenspiegel einzustellen
\usepackage[a4paper]{geometry}
%Mathepaket
\usepackage{amsmath}
%Symbole
\usepackage{amssymb}
%griechische Symbole
\usepackage{upgreek}
%weitere Symbole
\usepackage{pxfonts}
% Phonetischen Alphabete für LaTeX
\usepackage{tipa}
%farbige Schriften
\usepackage{xcolor}
\usepackage{scrhack}
%Bilder fixieren
\usepackage{float}
%Grafiken einbinden
\usepackage{graphicx}
% Kopf- und Fußzeilen
\usepackage[automark,standardstyle,markusedcase]{scrpage2}
% deutsche Überschriften
\usepackage[ngerman]{translator}
% Kopfzeilenabstand festlegen
\setlength{\headheight}{10mm}
%Abb. statt Abbildung
\usepackage{caption3}
%klickbare Referenzen
\usepackage{hyperref}
%draft Wasserzeichen
\usepackage{draftwatermark}
\SetWatermarkLightness{0.95}
\addto\captionsngerman{
\renewcommand{\figurename}{Abb.}
\renewcommand{\tablename}{Tab.}
}
%Glossar-Pakage
\usepackage[
nonumberlist, %keine Seitenzahlen anzeigen
acronym,      %ein Abkürzungsverzeichnis erstellen
toc]          %Einträge im Inhaltsverzeichnis      
{glossaries}
\usepackage{cite}
%Glossar einschalten
\makeglossaries



%Einstellungen Kopfzeile
\pagestyle{scrheadings} 
\setheadsepline{0.4pt}
\pagestyle{scrheadings}
\renewcommand*{\chapterpagestyle}{scrheadings}

%Zeilenabstand * 1.25 (default)
\renewcommand{\baselinestretch}{1.25}\normalsize
